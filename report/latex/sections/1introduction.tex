\chapter{Introduction}

Over the last few years, the energy supply and consumption in the Netherlands has changed drastically \citehere. 
For example, there has been an increase in energy generated by solar panels, as well as an increase in the number electric vehicles on roads \citehere. 
Both the solar panels on people's roofs as well the charging docks for their electric vehicles introduce higher harmonics into the electricity grid \citehere.
These higher harmonics increase the load on distribution transformers, which causes overheating, degraded performance and decreased lifetime \cite{vanDijk2022}

Distribution transformers are used to transform the high voltage electricity from the transmission grid to a lower voltage for the end-user.
In this process, some of the electrical potential energy is dissipated as heat. 

In modelling these losses, we can assume the magnetic field in the core to have the same temporal behavior as the the load on the transformer; sinusoidal with the same frequency. \citehere 
This works for load profiles with multiple frequencies, as long as the magnetic permeability of the core is assumed constant.

However, the magnetic permeability $\mu$ of the core is only constant by approximation. In reality $\mu$ depends on the magnetic field strength, which introduces non-linearities in the governing equations.
For low harmonics, i.e. $50$ Hz, these non-linearities are negligible.
For higher harmonics, the skin-effect causes the magnetic flux to be concentrated near the edges of the transformer. 
The flux in these edge regions is then high enough that $\mu$ can no longer be assumed to be constant \citehere. 
This results in a model error for higher harmonics.

Additionally, the superposition of the magnetic field resulting from several load frequencies is no longer valid for non-linear systems.
This is because each of the frequencies are now coupled to each other, through $mu$'s dependence on the magnetic field strength.

This paper aims to model the magnetic field in a distribution transformer, taking into account the non-linearities introduced by higher frequencies and loads.
The proposed model makes no assumptions on the frequency of the magnetic field resulting from a load profile containing (multiple) higher harmonics.

\Cref{sec:background} explains relevant background information. 
In \cref{sec:model}, the model is derived. 
Then, \cref{sec:results} presents the results of the model. 
Lastly, \cref{sec:discussion} discusses the results and the model.

