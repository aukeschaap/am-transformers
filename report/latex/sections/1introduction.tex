\chapter{Introduction}

Over the last few years, the energy supply and consumption in the Netherlands has changed drastically \citehere. For example, there has been an increase in energy generated by solar panels, but also an increase in the number electric vehicles on roads \citehere. These changes have an effect on the demand of electricity on the grid. More specific, they introduce higher harmonics into the alternating current. These higher harmonics cause an increased load on distribution transformers, which can lead to overheating and failure \citehere.

Distribution transformers are used to transform the high voltage electricity from the transmission grid to a lower voltage for the end-user. By magnetic induction, the voltage is transformed per the ratio of the number of windings on the primary and secondary side of the transformer. In this process, some of the energy is lost as heat. In modelling these losses, the typical approach is to assume that the current is periodic and sinusoidal \citehere. However, due to the introduction of higher harmonics, this assumption is no longer valid. Therefore, a new model is needed.

This paper aims to better model the losses of a distribution transformer, by providing a digital twin. The proposed model makes no assumptions on the frequencies of the introduced harmonics, by solving a time-dependent system of equations.

\Cref{sec:background} explains relevant background information. In \cref{sec:model}, the model is derived. Then, \cref{sec:results} presents the results of the model. Lastly, \cref{sec:discussion} discusses the results and the model.

