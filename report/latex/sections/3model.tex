\chapter{Model} \label{sec:model}


\section{Problem derivation}

To define the problem, we start with the Maxwell equations and corresponding constitutive relations. The Maxwell equations are given by
\begin{align*}
    \nabla \times \mathbf{E} &= -\frac{\partial \mathbf{B}}{\partial t}, \\
    \nabla \times \mathbf{H} &=  \mathbf{J} + \frac{\partial \mathbf{D}}{\partial t}, \\
    \nabla \cdot \mathbf{B} &= 0, \\
    \nabla \cdot \mathbf{D} &= \rho,
\end{align*}
where
\begin{itemize}
    \item $\mathbf{E}, [V/m]$ is the electric field intensity,
    \item $\mathbf{H}, [A/m]$ is the magnetic field intensity,
    \item $\mathbf{J}, [A/m^2]$ is the current density,
    \item $\mathbf{B}, [T]$ is the magnetic flux density,
    \item $\mathbf{D}, [C/m^2]$ is the electric flux density,
    \item $\rho, [C/m^3]$ is the free charge density.
\end{itemize}

\noindent These have the following constitutive relations:
\begin{align*}
    \mathbf{J} &= \mathbf{J_e} + \mathbf{J_c} \\
    \mathbf{B} &= \mu\mathbf{H} \\
    \mathbf{J}_c &= \sigma\mathbf E \\
    \mathbf{D} &= \epsilon \mathbf E
\end{align*}
where
\begin{itemize}
    \item $\mathbf{J_e}$ is the external current density,
    \item $\mathbf{J_c}$ is the conduction current density,
    \item $\sigma$ is the conductivity of the material,
    \item $\mu = \mu_0\mu_r$ is the permeability of the core,
    \item $\epsilon = \epsilon_0\epsilon_r$ is the permittivity of the material.
\end{itemize}

\begin{assumption}
    The permittivity of vacuum $\epsilon_0$ is very small, $(\mathcal{O}(10^{-12}))$, and for all materials in this research $\epsilon_r < 10$, so $\epsilon$ can be neclegted. Therefore, $\mathbf{D} = 0$ and can be neglected.
\end{assumption}

\noindent Substituting this in the Maxwell equations yields three equations,
\begin{align*}
    \nabla \times \mathbf{E} &= -\frac{\partial \mathbf{B}}{\partial t}, \\
    \nabla \times \left[\mu^{-1}\mathbf{B}\right] &=  \mathbf{J_e} + \sigma \mathbf{E}, \\
    \nabla \cdot \mathbf{B} &= 0.
\end{align*}

\noindent Using the potential formulation,
\begin{align*}
    \mathbf{E} &= -\nabla \varphi -\frac{\partial \mathbf{A}}{\partial t}, \\
    \mathbf{B} &= \nabla \times \mathbf A,
\end{align*}
we can formulate a system that we can solve.

\begin{assumption}
    We assume that the contribution of the electrostatic field $\varphi$ is negligble compared to the contribution of the potential field $\mathbf A$. That is, $\nabla \varphi = 0$, which implies that $\mathbf{E} = -\frac{\partial \mathbf{A}}{\partial t}$.
\end{assumption}

\begin{assumption}
    The flow of current is oriented along the $z$ axis, and the geometry is in the $xy$ plane. That is, $\mathbf{A} = (0, 0, A_z)$ and $\mathbf{J_e} = (0, 0, J_z)$. This implies $\nabla \times \mathbf A = \nabla A_z$.
\end{assumption}

\noindent Substituting in the Maxwell equations and rearranging, we obtain our problem definition.

\begin{problem}
    Find $A_z$ in the system
    \begin{equation}
        \sigma\frac{\partial A_z}{\partial t} = \nabla \times \left[\frac{1}{\mu}\nabla A_z\right] + J_z,
    \end{equation}
    where
    \begin{itemize}
        \item $A_z$ is the magnetic vector potential in the $z$ direction,
        \item $\mu$ is the permeability of the core,
        \item $J_z$ is the imposed source current density,
        \item $\sigma$ is the conductivity of the core.
    \end{itemize}
    From this point onwards, this will be formulated as
    \begin{equation}
        \sigma\dot u = \nabla \times \left[\frac{1}{\mu}\nabla u\right] + f.
    \end{equation}
\end{problem}

\newpage
\section{Finite Element Approach}
To solve this system, the finite element method can be used. This leads to the following weak form.
\begin{weakform}
    \begin{equation*}
        \sigma M \dot u = \frac{1}{\mu}K u + f,
    \end{equation*}
    where
    \begin{itemize}
        \item $u$ is $A_z$, the solution.
        \item $M$ is the mass matrix,
        \item $K$ is the stiffness matrix,
        \item $f$ is the source term, given by $J_z$,
    \end{itemize}
\end{weakform}
The solution of this problem is a time dependent solution, which makes it a difficult to solve problem. A few paths can be taken to simplify.