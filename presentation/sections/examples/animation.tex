
\section{Conclusion}
\begin{frame}[fragile]{animation}
  \vfill
  Some commands take optional arguments in the form of \verb|<x-y>|,
  where \verb|x| is the first `sub-frame' on which the context is shown,
  and \verb|y| is the last. \verb|x| or \verb|y| can be replaced by \verb|+|,
  referring to `the next sub-frame'. 
  \vfill
  \begin{columns}[onlytextwidth]
  \begin{column}{.5\textwidth}
    \begin{enumerate}
      \item<+-> uncovered\ldots
      \item<+-> one\ldots
      \item<+-> by\ldots
      \item<+-> one.
    \end{enumerate}
    \end{column}
  \begin{column}{.5\textwidth}
      Using only:\only<1>{1}\only<2>{2}\only<3>{3}

      Using onslide:\onslide<1>{1}\onslide<2>{2}\onslide<3>{3}

      Using pause:\pause1\pause2\pause3
  \end{column}
  \end{columns}
  \vfill
  For more advanced animations, see \S 14 of the manual:\\
  \url{https://www.ctan.org/pkg/beamer}
  % \url{https://www.ctan.org/pkg/animate}\\
  % \url{https://www.ctan.org/pkg/media9}
  \vfill
  % \transduration{2} automatic progression of slides
  \transpush<1>
\end{frame}